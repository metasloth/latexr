% Math 201 Homework 7
% ...so this is horribly/not formatted since it was written in about 10 minutes.

\documentclass[letterpaper,10pt]{article}
\usepackage[pass]{geometry}
\usepackage{commath}
\usepackage{amsfonts}
\usepackage{amsmath}
\linespread{1.25}
\begin{document}

\center{\bf MATHEMATICS 201-C (SPRING 2016): QUIZ 7}
\flushleft
~~1. (5 points) The following statement is either true or false. If the statement is
true, then prove it. If the statement is false, then disprove it.
\center{
For every natural number $n$, the integer $2n^{2}-4n+31$ is prime.
}
\flushleft
This is flase. \\
\emph{Disproof.} Let $n=31$. Note that $2(31)^{2}-4(31)+31=1829$, and that
$31\mid1829$. Hence, $2n^{2}-4n + 31$ is not prime when $n=31$, so
the statement is false.
\newline

~~2. (5 points) Prove the following statement with either induction, strong
induction or proof by smallest counterexample.
\center{
    \small
    For every integer $n \in \mathbb{N} $, it follows that $1^{2} + 2^{2} +
    3^{2} +\ldots+ n^{2} =$\(\displaystyle \frac{n(n + 1)(2n + 1)}{6}\)
}
\flushleft
\emph{Proof.} We will use mathematical induction to prove this. \\
~~~~First, let $n=1$. Note that $1^{2} =$
    \(\ \frac{(1)((1) + 1)(2(1) + 1)}{6} = 1\). Hence,
    $1^{2} + 2^{2} + 3^{2} +\ldots+ k^{2} =$\(\ \frac{k
    (k + 1)(2k + 1)}{6}\) is true for $k = 1$. \\
~~~~Next we will prove that for any integer $k \geq 1$, $S_{k} \implies S_{k+1}$.
    In other words, we mush show that if $1^{2} + 2^{2} + 3^{2} +\ldots+ k^{2} =$
    \(\ \frac{k(k + 1)(2k + 1)}{6}\) is true, then \center{ $1^{2} + 2^{2} + 3^{2} +\ldots+
    k^{2}+(k+1)^{2} =$ \(\displaystyle \frac{(k+1)((k+1) + 1)(2(k+1) + 1)}{6}\)}
    \flushleft is also true. Note that\[
        1^{2} + 2^{2} + 3^{2} + \ldots +k^{2}+(k+1)^{2} =
        (1^{2} + 2^{2} + 3^{2} +\ldots+k^{2})+(k+1)^{2}. \]
        If we substitute $S_k$ into this equation we now have
    \begin{align*}
        \frac{k(k + 1)(2k + 1)}{6} + (k+1)^{2} &=
        \frac{(k+1)((k+1) + 1)(2(k+1) + 1)}{6} \\
        ~ &= \frac{(k^{2}+3k+2)(2k+3)}{6} \\
        ~ &= \frac{2k^{3}+6k^{2}+4k+3k^{2}+9k+6}{6} \\
        ~ &= \frac{2k^{3}+3k^{2}+k}{6}+\frac{6k^{2}+12k+6}{6} \\
        ~ &= \frac{k(2k^{2}+3k+1)}{6}+\frac{6(k+1)^{2}}{6} \\
        ~ &= \frac{k(k+1)(2k+1)}{6}+(k+1)^{2}.
    \end{align*}
T~~~~herefore, $1^{2} + 2^{2} + 3^{2} +\ldots+
k^{2}+(k+1)^{2} = \frac{(k+1)((k+1) + 1)(2(k+1) + 1)}{6}$, which
means the statement is true for all $n=k+1$.



\end{document}
