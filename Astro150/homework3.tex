% Chapter 17 homework

\documentclass[letterpaper,11pt]{article}
\usepackage[margin=0.5in]{geometry}
\usepackage{commath}
\usepackage{amsfonts}
\usepackage{amsmath}

\pagestyle{empty}
\begin{document}
	
% Heading
\begin{center}
	\bf
	ASTRONOMY 150: HOMEWORK 3 - SAM HEMANN
\end{center}
	
	
	

% 26
\it
26. Explain the differences between the ways that hydrogen is
converted to helium in a low-mass star (proton-proton chain)
and in a high-mass star (CNO cycle). What is the catalyst in
the CNO cycle, and how does it take part in the reaction?
\normalfont
\smallskip
\par
Because the temperature at the center of a high mass star is significantly higher than
that of a lower mass star, the atoms in the center are moving fast 
enough for nuclear reactions with heavier elements. Unlike the proton-proton chain 
wherein hydrogen alone produces helium (by reacting with itself, then again with the
resulting isotopes), the CNO cycle utilizes carbon as a catalyst. As protons fuse
with heavier nuclei, carbon, heavier nitrogen and oxygen isotopes are created, 
eventually releasing helium and the starting carbon. Thus it is a more efficient 
reaction than the proton-proton chain since the "used", starting carbon remains afterwards. 


\bigskip
\it
% 27
27. How does a low-mass star begin burning helium in its core?
What about a high-mass star? How are these processes
different or similar?
\normalfont
\smallskip
\par
Unlike a low-mass star, the larger pressure in high-mass stars allows helium burning to begin before the core becomes degenerate. This allows for progressively heavier elements up to iron to burn throughout the star's life as a red supergiant, creating layers of shells burning heavier and heavier elements as the core collapses. This means there is immensely more energy at the end of a high-mass star's life compared to a low-mass star. 

\bigskip
\it
% 29
29. List the two reasons why each post-helium-burning cycle for
high-mass stars (carbon, neon, oxygen, silicon, and sulfur)
becomes shorter than the preceding cycle.
\normalfont
\smallskip
\par
1. The increasing pressure in the core means atoms are moving faster and faster, increasing the probability of collisions resulting in fusion.
\par
2. The reactions are not 100\% efficient, they emit energy, sometimes in the form of lighter elements at steps. These emitted atoms go on to fuel other reactions.

\bigskip
\it
% 30
30. Cepheids are highly luminous, variable stars in which the
period of variability is directly related to luminosity. Why are
Cepheids good indicators for determining stellar distances
that lie beyond the limits of accurate parallax measurements?
\normalfont
\smallskip
\par
Because cepheids expand and contract at a rate directly related to their luminosity, by measuring the length of the period we can deduce the luminosity of the star and compare this to the brightness we observe to derive how far away the star must be for it to appear as dim as it does.

\bigskip
\it
% 35
35. Why can the accretion disk around a neutron star release so
much more energy than the accretion disk around a white dwarf,
even though the two stars have approximately the same mass?
\normalfont
\smallskip
\par
The material in the accretion disk around a neutron star is generally rotating at a much smaller radius closer to the tiny neutron star, so by the law of angular momentum it spins much faster as well. This causes the material to heat up from kinetic energy, resulting in more energy released than that of slower, further away material around white dwarfs.

\bigskip
\it
% 38
38. Explain how astronomers know that there was an even
earlier generation of stars before the oldest observed stars.
\normalfont
\smallskip
\par
We know there are massive elements spread throughout the universe, and that only hydrogen and helium existed in the very first moments of the big bang. Thus there must have been prehistoric supernova that created these massive elements and "seeded" the universe, otherwise we ourselves couldn't  exist!
\bigskip
\bf
\begin{center}
	
	[Continued on back]
\end{center}

\pagebreak

\it
%50
50. In a large outburst in 1841, the 120-$M_\oplus$ star Eta Carinae was
losing mass at the rate of 0.1 $M_\oplus$ per year. Let’s put that into
perspective.
\par
a. The mass of the Sun is $2 \times 10^{30}$ kg. How much mass (in
kilograms) was Eta Carinae losing each minute?
\normalfont
\medskip
\par
\begin{center}
	$
	\displaystyle
	\frac{0.1M_\oplus}{1_{yr}} 
	\times 
	\frac{2\times10^{30} kg}{1M_\oplus } 
	\times
	\frac{1_{yr}}{525,600_{min}}
	=
	\frac{3.8\times10^{23} kg}{1 minute}
	$
\end{center}

\medskip
\par

b. The mass of the Moon is $7.35 \times 10^{22}$ kg. How does Eta Carinae’s
mass loss per minute compare with the mass of the Moon?
\normalfont
\begin{center}
	$
	\displaystyle
	\frac{0.1M_\oplus}{1_{yr}} 
	\times 
	\frac{2\times10^{30} kg}{1M_\oplus } 
	\times
	\frac{1 Moon}{7.35\times10^{22}}
	=
	\frac{2.7\times10^{6} Moons}{1_{yr}}
	$
\end{center}
	
\end{document}