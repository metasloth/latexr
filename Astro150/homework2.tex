% Chapter 13 homework

\documentclass[letterpaper,10pt]{article}
\usepackage[margin=0.5in]{geometry}
\usepackage{commath}
\usepackage{amsfonts}
\usepackage{amsmath}

\pagestyle{empty}
\begin{document}

% Heading
\begin{center}
    \bf
        ASTRONOMY 150: HOMEWORK 2 - SAM HEMANN
\end{center}



\par
% 29
\it
29. To know certain properties of a star, you must first determine
the star's distance. For other properties, knowledge of
distance is not necessary. Into which of these two categories
would you place each of the following properties: size, mass,
temperature, color, spectral type, and chemical composition?
In each case, state your reason(s).
\normalfont

\begin{quote}
None of these properties require us to know the distance to the
star:
\begin{description}
    \item[]
    Chemical composoition - We can look at a star's absorbtion
    and emission spectrums to determine the makeup
    \item[]
    Color - measuring the brightness of the star through red,
    blue, and green filters allows us to measure color.
    \item[]
    Temperature - this can be determined from the color via
    Wien's law or measurments of the spectral lines
    \item[]
    Size - the luminosity and temperature of a star can be used
    with Boltzmann's law, where energy is relative to the surface area
    \item[]
    Mass - for binary systems, we can use kepler's law once we
    observe the center of mass between the two stars, since the stars
    rotate around each other in a eliptical fashion. If we can't
    detect a center of mass we can look at the redshift and blushift
    of binary systems to derive the mass from the orbital period.
\end{description}
\end{quote}


\par
\it
% 31
31. In the constellation Cygnus, Albireo is a visual binary system
whose two components can be seen easily with even a small,
amateur telescope. Viewers describe the brighter star as "golden"
and the fainter one as "sapphire blue".
\begin{quote}
\begin{description}
    \item[]
    a. What does this description tell you about the relative
    temperatures of the two stars?
\normalfont
\par
	~~~~It suggests the brighter, golden star is cooler than
    the dimmer, sapphires star. This is because of the of the
    blackbody radiation of the stars, where more energy emmitted
    yields a more blueish light.
\par
\it
    \item[]
    b. What does it tell you about their respective sizes?
\normalfont
    \par
	~~~~The golden star is likely much larger than the sapphire one,
    because hotter stars emit more energy. Since the cooler, golden
    star is emmiting more light, it must have a lot more surface area
    than the sapphire star to appear brighter.
\end{description}
\end{quote}

\par
\it
% 32
32. Very cool stars have temperatures around 2500 K and emit
Planck spectra with peak wavelengths in the red part of the
spectrum. Do these stars emit any blue light? Explain your answer.
\normalfont
\begin{quote}
	~~~~Yes, although much less than yellow and red light. This is
    due to the blackbody curves of a star around 2500K increasing
    in brightness as the wavelength of light is longer.
\end{quote}


\par
\it
% 36
36. binary system has been determined, it can be assumed that
all other stars of the same spectral type and luminosity class
have the same mass. Why is this a safe assumption?
\normalfont
    \begin{quote}
	~~~~All stars are essentially massive fusion reactors adhering
    strictly to the same laws of physics. Since the spectral type and
    luminosity are directly related to the results of nuclear
    reactions, we can assume stars that share these properties
    are similar in mass.
\end{quote}


\par
\it
% 38
38. Very old stars often have very few heavy elements, while
very young stars have much more. What does this difference
imply about the chemical evolution of the universe?
\normalfont
\begin{quote}
    \par
	~~~~It certainly suggests a big-bang start to the universe,
    where only hydrogen and helium existed. As stars fused these
    into heavier atoms and dispersed them when dying, the overall
    ratio of heavier elements in the universe increased.
\end{quote}

\par
\it
% 51 (also find the distance to Betelgeus and Rigel in LY)
	% Hint: The parallax angle for Btgs is given in #50
51. Rigel (also in Orion) has a Hipparcos parallax of 0.00412
arcsec. Given that Betelgeuse and Rigel appear equally bright
in the sky, which star is actually more luminous? Knowing
that Betelgeuse appears reddish while Rigel appears bluish
white, which star would you say is larger and why? [Betelgeuse
has a paralax of 0.00763+-0.000164 arcsec]
\normalfont
\begin{quote}
    \begin{equation}
        d_{Rigel} ~ \alpha ~ \frac{1}{p} = \frac{1}{0.00412~arcsec} =
            242.718~pc = 791.640~ly
    \end{equation}
    \begin{equation}
        d_{Betelgeuse} ~ \alpha ~ \frac{1}{p} = \frac{1}{\approx0.00763~arcsec}
        \approx 118~pc \approx 385~ly
    \end{equation}
    ~~~~Because Rigel is about twice as far away, it's luminosity is cut in half
    (by the inverse squared law), so it must be twice as bright as Betelgeuse for
    the two stars to appear the same to us. Given that it is
    also hotter (bluer), their sizes may be pretty similar, since Rigel emits more
    light relative to its surface area.
\end{quote}


\end{document}
