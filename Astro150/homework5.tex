% Chapter 17 homework
\documentclass[letterpaper,11pt]{article}
\usepackage[margin=0.5in]{geometry}
\usepackage{commath}
\usepackage{amsfonts}
\usepackage{amsmath}
\pagestyle{empty}
\begin{document}



% Heading
\begin{center}
	\bf
	ASTRONOMY 150: HOMEWORK 5 - SAM HEMANN
\end{center}



% 34
\paragraph{34.}
\it
As the universe expands from the Big Bang, galaxies are not actually flying apart from one another. What is really happening?
\smallskip
	\par
	\normalfont
	Universal expansion refers to the space between objects increasing, but this expansion isn't from any central location. So galaxies like Andromeda can still move towards us since the gravitational pull between it and the Milky Way is much stronger than any effects of expansion.



% 35
\paragraph{35.}
\it
Knowing that you are studying astronomy, a curious friend asks where the center of the universe is located. You answer, “Right here and everywhere.” Explain in detail why you give this answer.
\smallskip
	\par
	\normalfont
	Because the laws of physics are exactly the same everywhere in the universe, phenomenon like universal expansion must hold true at every location. This means at any place in the universe, everything is expanding equally away from that particular place. This means we are each at the center of the entire universe from our perspective.



% 36
\paragraph{36.}
\it
The general relationship between recessional velocity ($v_r$) and redshift ($z$) is $v_r = cz$. This simple relationship fails, however, for very distant galaxies with large redshifts. Explain why.
\smallskip
	\par
	\normalfont
	A far away galaxy may have a redshift greater than one, which would imply the recessional velocity is faster than the speed of light, and this would be impossible. Thus $v_r = cz$ is only applicable when $z < 1$.



% 37
\paragraph{37.}
\it
Why is it significant that the CMB displays a Planck spectrum?
\smallskip
	\par
	\normalfont
	The Planck spectrum in the CMB offers strong evidence for a big bang start to the universe, since the observed radiation is what we would expect to see from a very hot, uniformly distributed hot gas. In essence it gives us a sort of rough "picture" of the start of our universe.



% 40
\paragraph{40.}
\it
Study Figure 19.6.
\smallskip
\par
a. Why is it important that the different "rungs of the distance ladder overlap in the distances that they measure?
\smallskip
	\par
	\normalfont
	It allows for greater precision and confidence in measuring the distances to objects - especially the "edge cases" of each method - since we can compare results found from the overlapping methods. For example, we can test the accuracy a new parallax-measuring instrument/system by comparing new data with values found using spectroscopic parallax.
	\par
	\bigskip
\par
\it
b. Why does the figure end at the right edge — because there are no more ways to measure distance or because there is no more universe to measure? How do you know?
\smallskip
	\par
	\normalfont
	The right edge represents the edge of our observable universe, so by definition we cannot measure (i.e. \textit{observe}) information beyond this. It shouldn't be thought of as a physical edge, but rather represent the furthest information has been able to travel relative to an observers perspective since the beginning of time. So even if you magically, instantaneously teleported to a location outside of Earth's observable universe, you would find yourself still at the center of an observable universe of the exact same "size".
	
	
	
\bigskip
\begin{center}
	\it
	(Continued on back)
\end{center}
\pagebreak



% 47
\paragraph{47.}
\it
The spectrum of a distant galaxy shows the $H\alpha$ line of hydrogen ($\lambda_{rest}$ = 656.28 nm) at a wavelength of 750 nm. Assume that $H_0$ = 70 km/s/Mpc.
\smallskip
\par
\it
a. What is the redshift ($z$) of this galaxy?
\smallskip
	\par
	\normalfont
	\begin{center}
		$
		\displaystyle
		z
		=
		\frac{\lambda_{obs}-\lambda_{rest}}{\lambda_{rest}}
		=
		\frac{750 \textrm{ nm} - 656.28 \textrm{ nm}}{656.28 \textrm{ nm}}
		=
		0.143
		$
	\end{center}
	\smallskip
\par
\it
b. What is its recessional velocity ($v_r$) in kilometers per second?
\smallskip
	\par
	\normalfont
	\begin{center}
		$
		\displaystyle
		v_r
		=
		cz
		=
		(299,792 \textrm{ km/s})(0.143)
		=
		42,870 \textrm{ km/s}
		$
	\end{center}
	\smallskip
\par
\it
c. What is the distance of the galaxy in megaparsecs?
\smallskip
	\par
	\normalfont
	\begin{center}
		$
		\displaystyle
		d
		=
		\frac{v_r}{H_0}
		=
		\frac{42,870 \textrm{ km/s}}{70 \textrm{ km/s/Mpc}}
		=
		612.4 \textrm{ Mpc}
		$
	\end{center}



% 50
\paragraph{50.}
\it
The temperature of the CMB is 2.73K. What is the peak wavelength of its Planck blackbody spectrum expressed both in microns and in millimeters?
\smallskip
	\par
	\normalfont
	\begin{center}
		$
		\displaystyle
		\lambda_{peak}
		=
		\frac{2.9\times10^6}{T}
		=
		\frac{2.9\times10^6}{2.73K}
		=
		1.06\times10^6 \textrm{ nm}
		=
		1.06\times10^3 \textrm{ microns}
		=
		1.06 \textrm{ mm}
		$
	\end{center}
	
\end{document}
